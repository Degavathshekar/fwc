%% Run LaTeX on this file several times to get Table of Contents,
%% cross-references, and citations.

\documentclass[11pt]{book}
\usepackage{gvv}
\usepackage{gvv-book-bkup}
%\usepackage{Wiley-AuthoringTemplate}
\usepackage[sectionbib,authoryear]{natbib}% for name-date citation comment the below line
%\usepackage[sectionbib,numbers]{natbib}% for numbered citation comment the above line

%%********************************************************************%%
%%       How many levels of section head would you like numbered?     %%
%% 0= no section numbers, 1= section, 2= subsection, 3= subsubsection %%
\setcounter{secnumdepth}{3}
%%********************************************************************%%
%%**********************************************************************%%
%%     How many levels of section head would you like to appear in the  %%
%%				Table of Contents?			%%
%% 0= chapter, 1= section, 2= subsection, 3= subsubsection titles.	%%
\setcounter{tocdepth}{2}
%%**********************************************************************%%

%\includeonly{ch01}
\makeindex

\begin{document}

\frontmatter
%%%%%%%%%%%%%%%%%%%%%%%%%%%%%%%%%%%%%%%%%%%%%%%%%%%%%%%%%%%%%%%%
%% Title Pages
%% Wiley will provide title and copyright page, but you can make
%% your own titlepages if you'd like anyway
%% Setting up title pages, type in the appropriate names here:

\booktitle{CBSE Math}

\subtitle{Made Simple}

\AuAff{G. V. V. Sharma}


%% \\ will start a new line.
%% You may add \affil{} for affiliation, ie,
%\authors{Robert M. Groves\\
%\affil{Universitat de les Illes Balears}
%Floyd J. Fowler, Jr.\\
%\affil{University of New Mexico}
%}

%% Print Half Title and Title Page:
%\halftitlepage
\titlepage

%%%%%%%%%%%%%%%%%%%%%%%%%%%%%%%%%%%%%%%%%%%%%%%%%%%%%%%%%%%%%%%%
%% Copyright Page

\begin{copyrightpage}{2023}
%Title, etc
\end{copyrightpage}

% Note, you must use \ to start indented lines, ie,
% 
% \begin{copyrightpage}{2004}
% Survey Methodology / Robert M. Groves . . . [et al.].
% \       p. cm.---(Wiley series in survey methodology)
% \    ``Wiley-Interscience."
% \    Includes bibliographical references and index.
% \    ISBN 0-471-48348-6 (pbk.)
% \    1. Surveys---Methodology.  2. Social 
% \  sciences---Research---Statistical methods.  I. Groves, Robert M.  II. %
% Series.\\

% HA31.2.S873 2004
% 001.4'33---dc22                                             2004044064
% \end{copyrightpage}

%%%%%%%%%%%%%%%%%%%%%%%%%%%%%%%%%%%%%%%%%%%%%%%%%%%%%%%%%%%%%%%%
%% Only Dedication (optional) 

%\dedication{To my parents}

\tableofcontents

%\listoffigures %optional
%\listoftables  %optional

%% or Contributor Page for edited books
%% before \tableofcontents

%%%%%%%%%%%%%%%%%%%%%%%%%%%%%%%%%%%%%%%%%%%%%%%%%%%%%%%%%%%%%%%%
%  Contributors Page for Edited Book
%%%%%%%%%%%%%%%%%%%%%%%%%%%%%%%%%%%%%%%%%%%%%%%%%%%%%%%%%%%%%%%%

% If your book has chapters written by different authors,
% you'll need a Contributors page.

% Use \begin{contributors}...\end{contributors} and
% then enter each author with the \name{} command, followed
% by the affiliation information.

% \begin{contributors}
% \name{Masayki Abe,} Fujitsu Laboratories Ltd., Fujitsu Limited, Atsugi, Japan
%
% \name{L. A. Akers,} Center for Solid State Electronics Research, Arizona State University, Tempe, Arizona
%
% \name{G. H. Bernstein,} Department of Electrical and Computer Engineering, University of Notre Dame, Notre Dame, South Bend, Indiana; formerly of
% Center for Solid State Electronics Research, Arizona
% State University, Tempe, Arizona 
% \end{contributors}

%%%%%%%%%%%%%%%%%%%%%%%%%%%%%%%%%%%%%%%%%%%%%%%%%%%%%%%%%%%%%%%%
% Optional Foreword:

%\begin{foreword}
%\lipsum[1-2]
%\end{foreword}

%%%%%%%%%%%%%%%%%%%%%%%%%%%%%%%%%%%%%%%%%%%%%%%%%%%%%%%%%%%%%%%%
% Optional Preface:

%\begin{preface}
%\lipsum[1-1]
%\prefaceauthor{}
%\where{place\\
% date}
%\end{preface}

% ie,
% \begin{preface}
% This is an example preface.
% \prefaceauthor{R. K. Watts}
% \where{Durham, North Carolina\\
% September, 2004}

%%%%%%%%%%%%%%%%%%%%%%%%%%%%%%%%%%%%%%%%%%%%%%%%%%%%%%%%%%%%%%%%
% Optional Acknowledgments:

%\acknowledgments
%\lipsum[1-2]
%\authorinitials{I. R. S.}  

%%%%%%%%%%%%%%%%%%%%%%%%%%%%%%%%
%% Glossary Type of Environment:

% \begin{glossary}
% \term{<term>}{<description>}
% \end{glossary}

%%%%%%%%%%%%%%%%%%%%%%%%%%%%%%%%
%\begin{acronyms}
%\acro{ASTA}{Arrivals See Time Averages}
%\acro{BHCA}{Busy Hour Call Attempts}
%\acro{BR}{Bandwidth Reservation}
%\acro{b.u.}{bandwidth unit(s)}
%\acro{CAC}{Call / Connection Admission Control}
%\acro{CBP}{Call Blocking Probability(-ies)}
%\acro{CCS}{Centum Call Seconds}
%\acro{CDTM}{Connection Dependent Threshold Model}
%\acro{CS}{Complete Sharing}
%\acro{DiffServ}{Differentiated Services}
%\acro{EMLM}{Erlang Multirate Loss Model}
%\acro{erl}{The Erlang unit of traffic-load}
%\acro{FIFO}{First in - First out}
%\acro{GB}{Global balance}
%\acro{GoS}{Grade of Service}
%\acro{ICT}{Information and Communication Technology}
%\acro{IntServ}{Integrated Services}
%\acro{IP}{Internet Protocol}
%\acro{ITU-T}{International Telecommunication Unit -- Standardization sector}
%\acro{LB}{Local balance}
%\acro{LHS}{Left hand side}
%\acro{LIFO}{Last in - First out}
%\acro{MMPP}{Markov Modulated Poisson Process}
%\acro{MPLS}{Multiple Protocol Labeling Switching}
%\acro{MRM}{Multi-Retry Model}
%\acro{MTM}{Multi-Threshold Model}
%\acro{PASTA}{Poisson Arrivals See Time Averages}
%\acro{PDF}{Probability Distribution Function}
%\acro{pdf}{probability density function}
%\acro{PFS}{Product Form Solution}
%\acro{QoS}{Quality of Service}
%\acro{r.v.}{random variable(s)}
%\acro{RED}{random early detection}
%\acro{RHS}{Right hand side}
%\acro{RLA}{Reduced Load Approximation}
%\acro{SIRO}{service in random order}
%\acro{SRM}{Single-Retry Model}
%\acro{STM}{Single-Threshold Model}
%\acro{TCP}{Transport Control Protocol}
%\acro{TH}{Threshold(s)}
%\acro{UDP}{User Datagram Protocol}
%\end{acronyms}

\setcounter{page}{1}

\begin{introduction}
This book links high school coordinate geometry to linear algebra and matrix analysis through solved problems.

\end{introduction}

\mainmatter
\chapter{Intersection of Conics}
\section{Chords }
\input{2022/chords.tex}
\section{Curves}
\input{2022/curves.tex}
\chapter{Tangent And Normal}
\input{2022/tangent.tex}
\section{Construction}
%\iffalse
\chapter{Vectors}
\section{projection vectors}
\input{projections.tex}
\section{product vectors}
\begin{enumerate}
\item $\overrightarrow{a}$   and  $\overrightarrow{ b}$are two unit vectors such that  $\abs { 2\overrightarrow{ a}+3\overrightarrow{ b}}$ = $\abs{3\overrightarrow{ a} - 2\overrightarrow{ b}}$. Find the angle between $\overrightarrow{ a }$ and $\overrightarrow{ b }$.
\item If $\overrightarrow{ a}$  and $\overrightarrow{b}$ are two vectors such that  $\overrightarrow{a} = \hat{i} - \hat{j} + \hat{k}$   and  $\overrightarrow{b} =  2\hat{i} - \hat{j} - 3\hat{k}$ then find the vector $\overrightarrow{c}$, given that $\overrightarrow{a} \times \overrightarrow{c} = \overrightarrow{b}$   and $\overrightarrow{a}.\overrightarrow{c}$= 4.
\item  If $\abs{\overrightarrow{ a } \times \overrightarrow { b }}^2 + \abs { \overrightarrow{ a } . \overrightarrow{ b }}^2$= 400 and  $\abs { \overrightarrow{ b}} = 5$, find the value of  $\abs{\overrightarrow{ a }}$.
\item If $\overrightarrow{a} = \hat{i} + \hat{ j} + \hat{ k} , \overrightarrow{a} . \overrightarrow{b}$ = 1  and $\overrightarrow{a} \times \overrightarrow{b} = \hat{j} - \hat{k}$,  then find  $\abs{\overrightarrow{b}}$
\item If $\abs{\overrightarrow{ a}}= 3, \abs{\overrightarrow{ b}} = 2\sqrt{ 3}$  and $\overrightarrow{ a} . \overrightarrow{ b}$ = 6,  then find the value of $\abs{\overrightarrow{ a} \times \overrightarrow{ b}}$.	
\item $\abs{\overrightarrow{a}} = 8, \abs{\overrightarrow{ b}} = 3$ and $\overrightarrow{a} . \overrightarrow{b} = 12\sqrt{3}$, then the value of  $\abs{\overrightarrow{a} \times \overrightarrow{b}}$  is                                     
\begin{enumerate}                                      
\item  24 
\item  144
\item  2 
\item  12 
\end{enumerate}
If$\space$ $\overrightarrow{ a} = 2\hat{i} + \hat{j} + 3\hat{k}, \hat{b} = -\hat{i} + 2\hat{j} + \hat{k}$ and $\overrightarrow{c} = 3\hat{i} + \hat{j} + 2\hat{k}$, then find $\overrightarrow{a} . (\overrightarrow{ b} \times \overrightarrow{c})$.
\item $\overrightarrrow{a}, \overrightarrow{ b },\overrightarrow{ c }$  and  $\overrightarrow{ d }$ are four non-zeros vectors such that  $\overrightarrow{a}\times \overrightarrow{b}= \overrightarrow{c} \times \overrightarrow{d}$  and  $\overrightarrow{a} \times \overrightarrow{c} = 4\overrightarrow{b} \times \overrightarrow{d}$, then show that  $(\overrightarrow{ a}-2\overrightarrow{d} \text{ is parallel to}(2\overrightarrow{b}-\overrightarrow{c})$ where $\overrightarrow{a} \neq 2\overrightarrow{d}, \overrightarrow{c} \neq 2\overrightarrow{b}$
\item If $\overrightarrow{a} = \hat{i} + \hat{ j} + \hat{ k} , \overrightarrow{a} . \overrightarrow{b} = 1 \text{ and} \overrightarrow{a} \times \overrightarrow{b} =\hat{j} - \hat{k}, \text { then find } \abs{\overrightarrow{b}}$
\item (a) If $\overrightarrow{ a}$  and  $\overrightarrow{b}$  are two vectors such that $\abs{\overrightarrow{a} + \overrightarrow{b}} = \abs{ \overrightarrow{b}}$,then prove that $(\overrightarrow{a} + 2\overrightarrow{b})$  is perpendicular to $\overrightarrow{ a}$.
\item If $\overrightarrow{ a}$ and $\overrightarrow{ b}$ are unit vectors and $\theta$ is the angle between them , then prove that sin $\dfrac{\theta}{ 2} = \dfrac{1}{2}\abs{\overrightarrow{ a} - \overrightarrow{ b}}$.
\item If $\overrightarrow{a}$ and $\overrightarrow{b}$  are two unit vectors such that and $\theta$ is the angle between them, then prove that                       
\begin{center}                                         
sin $\dfrac{ \theta}{2} = \dfrac{1}{2} \abs{\overrightarrow{a} - \overrightarrow{b}}$\\                       
\end{center}
\end{enumerate}


\section{position vectors}
\begin{enumerate}
\item If $\overrightarrow{a} , \overrightarrow{b}$ and $\overrightarrow{c}$ are the position vectors of the points A(2, 3, -4), B(3, -4, -5) and C(3, 2,-3) and respectively, then $\abs{\overrightarrow{a} + \overrightarrow{b} + \overrightarrow{c}}$ is equal to                
\begin{enumerate}                                      
\item $\sqrt{113}$                                     
\item $\sqrt{185}$                                     
\item $\sqrt{203}$                                    
\item $\sqrt{209}$                               
\end{enumerate}
\end{enumerate}

\section{Section Formula}
\begin{enumerate}
\item A circle has its center at (4,4). If one end of adiameter is (4,0), then find the coordinates of other end.
\end{enumerate}

\section{plane vectors}
\begin{enumerate}
\item Find the values $\lambda$, for which the distanceof point ( 2,1, $\lambda$) from plane $3x+5y+4z=11$ is  $2\sqrt{2}$ units.
\item Find the coordinates of the point where the line through (3,4,1) crosses the ZX-plane
\end{enumerate}

\section{geometry vectors}
\begin{enumerate}
\item Using vectors, find the area of the triangle withvertices A(-1, 0, -2), B(0, 2, 1) and C(-1, 4,1)
\item Using integration, find the area of triangle region whose vertices are (2,0) , (4,5) and (1,4).
\end{enumerate}

\section{Distance formula}
\input{distanceformula.tex}
\section{Direction vectors}
\begin{enumerate}
\item (a) If a line makes $60\degree$  and $45\degree$ angles with the positive directions of X-axis and z-axis respectively, then find the angle that it makes with the positive direction of y-axis. Hence, write the direction cosines of the line.
\item The Cartesian equation of a line AB is :       
\begin{center}
$\dfrac{2x-1}{12} = \dfrac{ y+2}{2} = \dfrac{z-3}{3}$.\\
\end{center}                                      
\item Find the directions cosines of a line parallel to line AB.
\item Find the direction cosines of a line whose cartesian equation is given as 3x + 1 = 6y - 2 = 1 - z.
\item A vector of magnitude 9 units in the direction ofthe vector $-2\hat{i} - \hat{j} + 2\hat{k}$ is \underline{\hspace{1cm}}    
\end{enumerate}

\section{Diagonal vectors}
\begin{enumerate}
\item The two adajacent sides of a parallelogram are represented by $2\hat{i}-4\hat{j}-5\hat{k}$ and $\hat{ i}+2\hat{j}+3\hat{k}$. Find the unit vectors parallel toits diagonals. Using the diagonal vectors, find the area ofthe parallelogram also.
\item The two adjacent sides of a parallelogram are represented by vectors $2\hat{i} - 4\hat{j} + 5\hat{k}$  and  $\hat{ i} - 2\hat{j} - 3\hat{k}$. Find the unit vector parallel to one of its diagonals. Also,find the area of the parallelogram.
\item If $\space$ $\overrightarrow{ a} = \overrightarrow{i} + 2\overrightarrow{j} + 3\overrightarrow{k} \text { and} \overrightarrow{ b} = 2\hat{i} + 4\hat{j} - 5\hat{k}$ represent two adjacent sides of a parallelogram, then find the unit vector parallel to the diagonal of the parallelogram
\end{enumerate}

\section{Area of triangle}
\begin{enumerate}
\item  Find the area of the quadrilateral ABCD whose vertices are A(-4, -3) , B(3, -1), C(0, 5) and D(-4, 2)  
\item If the points A(2,0), B(6,1), and c(p ,q) form a triangle of area 12sq. units (positive only) and 2p + q = 10,then find the values of p and q.                 
\end{enumerate}

%%\section{Exercises}
%\input{chapters/vectors/exer/ortho.tex}
%\section{Vector Product}
%\input{chapters/vectors/examples/cross.tex}
%\section{Exercises}
%\input{chapters/vectors/exer/cross.tex}
%\section{Miscellaneous}
%\input{chapters/vectors/examples/misc.tex}
%\section{Exercises}
%\input{chapters/vectors/exer/misc.tex}
%\section{Triangle}
%\input{chapters/const/examples/tri.tex}
%\section{Exercises}
%\input{chapters/const/exer/tri.tex}
%\section{ Quadrilateral}
%\input{chapters/const/examples/quad.tex}
%\section{Exercises}
%\input{chapters/const/exer/quad.tex}
\iffalse
\chapter{Linear Forms}
\section{Equation of a Line}
\input{chapters/linear/examples/equation.tex}
\section{Perpendicular}
\input{chapters/linear/examples/perp.tex}
\section{Plane}
\input{chapters/linear/examples/plane.tex}
\section{Miscellaneous }
\input{chapters/linear/examples/misc.tex}
\section{Exemplar}
\input{exemplar/11.10.3}
\section{Singular Value Decomposition}
\input{svd/svd.tex}
%
%\chapter{Constructions}
%\section{JEE}
%\input{jee/7.tex}
%--------------------------------------------------------
%\section{Properties}
%\input{chapters/9/9/3.tex}
%
%\section{Properties}
%\input{chapters/9/8/1.tex}
%\section{Mid Point Theorem}
%\input{chapters/9/8/2.tex}
%\section{Parallelograms}
%\section{Triangles and Parallelograms}
%\input{chapters/9/9/4.tex}
%--------------------------------------------------------

%\chapter{Circles}
%
%\chapter{Tangents to a Circle}


\chapter{Conics}
\section{Circle}
\input{chapters/circles/examples/equation.tex}
\section{Exercises}
\input{chapters/circles/exer/equation.tex}
\section{Construction}
\input{chapters/circles/examples/const.tex}
\section{Exercises}
\input{chapters/circles/exer/const.tex}
\section{Parabola}
\input{chapters/conics/examples/parab.tex}
\section{Exercises}
\input{chapters/conics/exer/parab.tex}
\section{Ellipse}
\input{chapters/conics/examples/ellipse.tex}
\section{Exercises}
\input{chapters/conics/exer/ellipse.tex}
\section{Hyperbola}
\input{chapters/conics/examples/hyper.tex}
\section{Exercises}
\input{chapters/conics/exer/hyper.tex}

%
\fi


%\include{ch02} 
\backmatter
\appendix
\iffalse
\chapter{ Vectors}
\section{$2\times 1$ vectors}
\input{matrix/two.tex}
%\include{app01}
%\appendix
\section{$3\times 1$ vectors}
\input{matrix/three.tex}
\chapter{Matrices}
\input{matrix/mat.tex}
\input{linman/chapters/decomp/svd.tex}

\chapter{Triangle Constructions}
\input{cons/tri.tex}


\chapter{Linear Forms}
\section{Two Dimensions}
\input{linear/two.tex}
\section{Three Dimensions}
\input{linear/three.tex}
\chapter{Quadratic Forms}
%\numberwithin{equation}{subsection}
%\numberwithin{equation}{section}
\section{Conic equation }
\input{quad/defs.tex}
\section{Circles}
\input{quad/circle.tex}

\section{Standard Form}
\input{quad/stddef.tex}
\chapter{Conic Parameters}
\section{Standard Form}
\input{quad/standard.tex}
\section{Quadratic Form }
\input{quad/coroll.tex}

\chapter{Conic Lines}
\section{Pair of Straight Lines}
%
\input{quad/pair.tex}
\section{Intersection of Conics}
\input{quadlines/inter.tex}
\section{ Chords of a Conic}
\input{quadlines/chord.tex}
\section{ Tangent and Normal}
\input{quadlines/tangent.tex}
\fi
%\chapter{Proofs}
%   \section{}
%\input{apps/defs.tex}

%  \section{}
%\input{apps/parab.tex}
%  \section{}
%\input{apps/nonparab.tex}
%		\section{}
%\input{apps/params.tex}
\latexprintindex

\end{document}

 
\section{Examples}
\subsection{Loney}
\input{examples/loney.tex}
\subsection{Miscellaneous}
\input{examples/misc.tex}
%
%%\section*{Disclosure Statement}
%%The authors report there are no competing interests to declare.
%%
%%
%%
%%  
%%%All the results related to conics are summarized in 
%%%Table \ref{table:conics}.  
%%%\begin{table*}[!t]
%%%\centering
%%%\input{conics.tex}
%%%%\input{./figs/conics.tex}
%%%\caption{$\vec{x}^{\top}\vec{V}\vec{x}+2\vec{u}^{\top}\vec{x}+f = 0$  can be expressed in the above standard form for various conics. $\vec{c}$ represents the centre/vertex of the conic. $\vec{q}$ is/are the point(s) of contact for the tangent(s). }
%%%\label{table:conics}
%%%\end{table*}
%%%\begin{verbatim}
%%\bibliographystyle{tfs}
%%%\bibliography{interacttfssample}
%%\bibliography{school}
%%\end{verbatim}
%% included where the list of references is to appear, where \texttt{tfs.bst} is the name of the \textsc{Bib}\TeX\ bibliography style file for Taylor \& Francis' Reference Style S and \texttt{interacttfssample.bib} is the bibliographic database included with the \textsf{Interact}-TFS \LaTeX\ bundle (to be replaced with the name of your own .bib file). \LaTeX/\textsc{Bib}\TeX\ will extract from your .bib file only those references that are cited in your .tex file and list them in the References section.
%
%% Please include a copy of your .bib file and/or the final generated .bbl file among your source files if your .tex file does not contain a reference list in a \texttt{thebibliography} environment.
%

  % \section{Appendices}
  % \appendix
			\appendices
